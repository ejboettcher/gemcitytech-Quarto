% Options for packages loaded elsewhere
\PassOptionsToPackage{unicode}{hyperref}
\PassOptionsToPackage{hyphens}{url}
\PassOptionsToPackage{dvipsnames,svgnames,x11names}{xcolor}
%
\documentclass[
]{report}

% fancy header
\usepackage{geometry}
\usepackage{fancyhdr}
\pagestyle{fancy}
\fancypagestyle{custom-fancy}{%
\fancyhf{}
\fancyhfoffset[L]{1cm} % left extra length
\fancyhfoffset[R]{1cm} % right extra length

    \chead{\textcolor{orange}{\textbf{Fun}}}
    \cfoot{\textcolor{orange}{\textbf{Fun}}}
\rfoot{}
}
\fancypagestyle{plain}{%
\fancyhf{}
\fancyhfoffset[L]{1cm} % left extra length
\fancyhfoffset[R]{1cm} % right extra length
\rhead{\today}

    \chead{\textcolor{orange}{\textbf{Fun}}}
    \cfoot{\textcolor{orange}{\textbf{Fun}}}
\rfoot{\thepage}
}

\addtocontents{toc}{\protect\thispagestyle{custom-fancy}}


\usepackage{amsmath,amssymb}
\usepackage{iftex}
\ifPDFTeX
  \usepackage[T1]{fontenc}
  \usepackage[utf8]{inputenc}
  \usepackage{textcomp} % provide euro and other symbols
\else % if luatex or xetex
  \usepackage{unicode-math}
  \defaultfontfeatures{Scale=MatchLowercase}
  \defaultfontfeatures[\rmfamily]{Ligatures=TeX,Scale=1}
\fi
\usepackage{lmodern}
\ifPDFTeX\else  
    % xetex/luatex font selection
\fi
% Use upquote if available, for straight quotes in verbatim environments
\IfFileExists{upquote.sty}{\usepackage{upquote}}{}
\IfFileExists{microtype.sty}{% use microtype if available
  \usepackage[]{microtype}
  \UseMicrotypeSet[protrusion]{basicmath} % disable protrusion for tt fonts
}{}
\makeatletter
\@ifundefined{KOMAClassName}{% if non-KOMA class
  \IfFileExists{parskip.sty}{%
    \usepackage{parskip}
  }{% else
    \setlength{\parindent}{0pt}
    \setlength{\parskip}{6pt plus 2pt minus 1pt}}
}{% if KOMA class
  \KOMAoptions{parskip=half}}
\makeatother
\usepackage{xcolor}
\setlength{\emergencystretch}{3em} % prevent overfull lines
\setcounter{secnumdepth}{-\maxdimen} % remove section numbering
% Make \paragraph and \subparagraph free-standing
\ifx\paragraph\undefined\else
  \let\oldparagraph\paragraph
  \renewcommand{\paragraph}[1]{\oldparagraph{#1}\mbox{}}
\fi
\ifx\subparagraph\undefined\else
  \let\oldsubparagraph\subparagraph
  \renewcommand{\subparagraph}[1]{\oldsubparagraph{#1}\mbox{}}
\fi

\usepackage{color}
\usepackage{fancyvrb}
\newcommand{\VerbBar}{|}
\newcommand{\VERB}{\Verb[commandchars=\\\{\}]}
\DefineVerbatimEnvironment{Highlighting}{Verbatim}{commandchars=\\\{\}}
% Add ',fontsize=\small' for more characters per line
\usepackage{framed}
\definecolor{shadecolor}{RGB}{241,243,245}
\newenvironment{Shaded}{\begin{snugshade}}{\end{snugshade}}
\newcommand{\AlertTok}[1]{\textcolor[rgb]{0.68,0.00,0.00}{#1}}
\newcommand{\AnnotationTok}[1]{\textcolor[rgb]{0.37,0.37,0.37}{#1}}
\newcommand{\AttributeTok}[1]{\textcolor[rgb]{0.40,0.45,0.13}{#1}}
\newcommand{\BaseNTok}[1]{\textcolor[rgb]{0.68,0.00,0.00}{#1}}
\newcommand{\BuiltInTok}[1]{\textcolor[rgb]{0.00,0.23,0.31}{#1}}
\newcommand{\CharTok}[1]{\textcolor[rgb]{0.13,0.47,0.30}{#1}}
\newcommand{\CommentTok}[1]{\textcolor[rgb]{0.37,0.37,0.37}{#1}}
\newcommand{\CommentVarTok}[1]{\textcolor[rgb]{0.37,0.37,0.37}{\textit{#1}}}
\newcommand{\ConstantTok}[1]{\textcolor[rgb]{0.56,0.35,0.01}{#1}}
\newcommand{\ControlFlowTok}[1]{\textcolor[rgb]{0.00,0.23,0.31}{#1}}
\newcommand{\DataTypeTok}[1]{\textcolor[rgb]{0.68,0.00,0.00}{#1}}
\newcommand{\DecValTok}[1]{\textcolor[rgb]{0.68,0.00,0.00}{#1}}
\newcommand{\DocumentationTok}[1]{\textcolor[rgb]{0.37,0.37,0.37}{\textit{#1}}}
\newcommand{\ErrorTok}[1]{\textcolor[rgb]{0.68,0.00,0.00}{#1}}
\newcommand{\ExtensionTok}[1]{\textcolor[rgb]{0.00,0.23,0.31}{#1}}
\newcommand{\FloatTok}[1]{\textcolor[rgb]{0.68,0.00,0.00}{#1}}
\newcommand{\FunctionTok}[1]{\textcolor[rgb]{0.28,0.35,0.67}{#1}}
\newcommand{\ImportTok}[1]{\textcolor[rgb]{0.00,0.46,0.62}{#1}}
\newcommand{\InformationTok}[1]{\textcolor[rgb]{0.37,0.37,0.37}{#1}}
\newcommand{\KeywordTok}[1]{\textcolor[rgb]{0.00,0.23,0.31}{#1}}
\newcommand{\NormalTok}[1]{\textcolor[rgb]{0.00,0.23,0.31}{#1}}
\newcommand{\OperatorTok}[1]{\textcolor[rgb]{0.37,0.37,0.37}{#1}}
\newcommand{\OtherTok}[1]{\textcolor[rgb]{0.00,0.23,0.31}{#1}}
\newcommand{\PreprocessorTok}[1]{\textcolor[rgb]{0.68,0.00,0.00}{#1}}
\newcommand{\RegionMarkerTok}[1]{\textcolor[rgb]{0.00,0.23,0.31}{#1}}
\newcommand{\SpecialCharTok}[1]{\textcolor[rgb]{0.37,0.37,0.37}{#1}}
\newcommand{\SpecialStringTok}[1]{\textcolor[rgb]{0.13,0.47,0.30}{#1}}
\newcommand{\StringTok}[1]{\textcolor[rgb]{0.13,0.47,0.30}{#1}}
\newcommand{\VariableTok}[1]{\textcolor[rgb]{0.07,0.07,0.07}{#1}}
\newcommand{\VerbatimStringTok}[1]{\textcolor[rgb]{0.13,0.47,0.30}{#1}}
\newcommand{\WarningTok}[1]{\textcolor[rgb]{0.37,0.37,0.37}{\textit{#1}}}

\providecommand{\tightlist}{%
  \setlength{\itemsep}{0pt}\setlength{\parskip}{0pt}}\usepackage{longtable,booktabs,array}
\usepackage{calc} % for calculating minipage widths
% Correct order of tables after \paragraph or \subparagraph
\usepackage{etoolbox}
\makeatletter
\patchcmd\longtable{\par}{\if@noskipsec\mbox{}\fi\par}{}{}
\makeatother
% Allow footnotes in longtable head/foot
\IfFileExists{footnotehyper.sty}{\usepackage{footnotehyper}}{\usepackage{footnote}}
\makesavenoteenv{longtable}
\usepackage{graphicx}
\makeatletter
\def\maxwidth{\ifdim\Gin@nat@width>\linewidth\linewidth\else\Gin@nat@width\fi}
\def\maxheight{\ifdim\Gin@nat@height>\textheight\textheight\else\Gin@nat@height\fi}
\makeatother
% Scale images if necessary, so that they will not overflow the page
% margins by default, and it is still possible to overwrite the defaults
% using explicit options in \includegraphics[width, height, ...]{}
\setkeys{Gin}{width=\maxwidth,height=\maxheight,keepaspectratio}
% Set default figure placement to htbp
\makeatletter
\def\fps@figure{htbp}
\makeatother

\makeatletter
\makeatother
\makeatletter
\makeatother
\makeatletter
\@ifpackageloaded{caption}{}{\usepackage{caption}}
\AtBeginDocument{%
\ifdefined\contentsname
  \renewcommand*\contentsname{Table of contents}
\else
  \newcommand\contentsname{Table of contents}
\fi
\ifdefined\listfigurename
  \renewcommand*\listfigurename{List of Figures}
\else
  \newcommand\listfigurename{List of Figures}
\fi
\ifdefined\listtablename
  \renewcommand*\listtablename{List of Tables}
\else
  \newcommand\listtablename{List of Tables}
\fi
\ifdefined\figurename
  \renewcommand*\figurename{Figure}
\else
  \newcommand\figurename{Figure}
\fi
\ifdefined\tablename
  \renewcommand*\tablename{Table}
\else
  \newcommand\tablename{Table}
\fi
}
\@ifpackageloaded{float}{}{\usepackage{float}}
\floatstyle{ruled}
\@ifundefined{c@chapter}{\newfloat{codelisting}{h}{lop}}{\newfloat{codelisting}{h}{lop}[chapter]}
\floatname{codelisting}{Listing}
\newcommand*\listoflistings{\listof{codelisting}{List of Listings}}
\makeatother
\makeatletter
\@ifpackageloaded{caption}{}{\usepackage{caption}}
\@ifpackageloaded{subcaption}{}{\usepackage{subcaption}}
\makeatother
\makeatletter
\@ifpackageloaded{tcolorbox}{}{\usepackage[skins,breakable]{tcolorbox}}
\makeatother
\makeatletter
\@ifundefined{shadecolor}{\definecolor{shadecolor}{rgb}{.97, .97, .97}}
\makeatother
\makeatletter
\makeatother
\makeatletter
\makeatother
\ifLuaTeX
  \usepackage{selnolig}  % disable illegal ligatures
\fi
\IfFileExists{bookmark.sty}{\usepackage{bookmark}}{\usepackage{hyperref}}
\IfFileExists{xurl.sty}{\usepackage{xurl}}{} % add URL line breaks if available
\urlstyle{same} % disable monospaced font for URLs
\hypersetup{
  pdftitle={Custom Template},
  pdfauthor={Evelyn J. Boettcher},
  colorlinks=true,
  linkcolor={blue},
  filecolor={Maroon},
  citecolor={Blue},
  urlcolor={Blue},
  pdfcreator={LaTeX via pandoc}}

\title{Custom Template}
\usepackage{etoolbox}
\makeatletter
\providecommand{\subtitle}[1]{% add subtitle to \maketitle
  \apptocmd{\@title}{\par {\large #1 \par}}{}{}
}
\makeatother
\subtitle{python 101}
\author{Evelyn J. Boettcher}
\date{2024-09-08}

\begin{document}
 
  \begin{titlepage}
  
  \begin{titlepage}
  \thispagestyle{custom-fancy}
  \vspace{2cm}

  \begin{center}
    
    \Huge
    \textbf{Custom Template}
    \vspace{1cm}
   
    \huge
        python 101
      \vspace{1cm}
    \end{center}


  \vfill

  \Large
  Author: Evelyn J. Boettcher \\
  Date: 2024-09-08



  % this puts the classification authority info on the front page
  \fbox{
  \begin{minipage}[c][\totalheight][b]{3in}
    \begin{tabular}[b]{r@{\em:\em}l}
      No one, ~I have no authority \\
      Classified on 2024-09-08
    \end{tabular}
  \end{minipage}}
  \end{titlepage}

  \title{Custom Template}  \end{titlepage}


\thispagestyle{plain} 

\ifdefined\Shaded\renewenvironment{Shaded}{\begin{tcolorbox}[boxrule=0pt, borderline west={3pt}{0pt}{shadecolor}, frame hidden, interior hidden, breakable, sharp corners, enhanced]}{\end{tcolorbox}}\fi

\renewcommand*\contentsname{Table of contents}
{
\hypersetup{linkcolor=}
\setcounter{tocdepth}{2}
\tableofcontents
}
\hypertarget{section}{%
\chapter{}\label{section}}

\hypertarget{week-3-lesson-1-linear-regression}{%
\section{Week 3 Lesson 1: Linear
Regression}\label{week-3-lesson-1-linear-regression}}

\hypertarget{problem-1}{%
\subsection{Problem 1}\label{problem-1}}

\begin{itemize}
\tightlist
\item
  Explain what np.polynomial does in this script. ** And what are my
  options besides Polynomial
\end{itemize}

\begin{Shaded}
\begin{Highlighting}[]
\ImportTok{import}\NormalTok{ numpy }\ImportTok{as}\NormalTok{ np}
\ImportTok{import}\NormalTok{ pandas }\ImportTok{as}\NormalTok{ pd}

\NormalTok{df }\OperatorTok{=}\NormalTok{ pd.read\_csv(}\StringTok{\textquotesingle{}../data/co2\_weekly\_mlo.txt\textquotesingle{}}\NormalTok{, skiprows}\OperatorTok{=}\DecValTok{49}\NormalTok{,}
\NormalTok{                     names}\OperatorTok{=}\NormalTok{[}\StringTok{\textquotesingle{}yr\textquotesingle{}}\NormalTok{, }\StringTok{\textquotesingle{}mon\textquotesingle{}}\NormalTok{, }\StringTok{\textquotesingle{}day\textquotesingle{}}\NormalTok{, }\StringTok{\textquotesingle{}decimal\textquotesingle{}}\NormalTok{, }\StringTok{\textquotesingle{}ppm\textquotesingle{}}\NormalTok{, }\StringTok{\textquotesingle{} \#days\textquotesingle{}}\NormalTok{, }\StringTok{\textquotesingle{}1 yr ago\textquotesingle{}}\NormalTok{, }\StringTok{\textquotesingle{}10 yr ago\textquotesingle{}}\NormalTok{, }\StringTok{\textquotesingle{}since 1800\textquotesingle{}}\NormalTok{],}
\NormalTok{                     delim\_whitespace}\OperatorTok{=}\VariableTok{True}\NormalTok{)}
\NormalTok{clean\_df }\OperatorTok{=}\NormalTok{ df[df.ppm }\OperatorTok{!=} \OperatorTok{{-}}\FloatTok{999.99}\NormalTok{]}
\NormalTok{pp }\OperatorTok{=}\NormalTok{ np.polynomial.Polynomial(np.polyfit(clean\_df.decimal, clean\_df.ppm, }\DecValTok{1}\NormalTok{))}
\NormalTok{plt.scatter(clean\_df.decimal, clean\_df.ppm)}
\NormalTok{plt.plot(clean\_df.decimal, pp(clean\_df.decimal), color}\OperatorTok{=}\StringTok{\textquotesingle{}red\textquotesingle{}}\NormalTok{)}
\NormalTok{plt.show()}
\end{Highlighting}
\end{Shaded}

\hypertarget{discover-computer-science-teachable-machine-workshop}{%
\section{Discover Computer Science: Teachable Machine
Workshop}\label{discover-computer-science-teachable-machine-workshop}}

A No / Low Code workshop where students will learn about machine
learning (ML) and \textbf{build} their own ML application.

\begin{itemize}
\tightlist
\item
  Training the AI/ML model is a NO code exercise.\\
\item
  Creating a working web application is a \emph{low} code exercise.

  \begin{itemize}
  \tightlist
  \item
    Students will modify a working application for their needs.
  \end{itemize}
\end{itemize}

\hypertarget{targeted-grades}{%
\subsection{Targeted Grades}\label{targeted-grades}}

4th through 12th

This mainly targets to middle school to elementary. But there is no age
limit on this workshop.

\hypertarget{slide-deck.}{%
\subsection{Slide Deck.}\label{slide-deck.}}

\href{https://ejboettcher.github.io/gemcityML-No-CodeAI/slide_deck.html\#/title-slide}{Slide
Deck}

\hypertarget{duration}{%
\subsection{Duration}\label{duration}}

60-90 minutes

\hypertarget{outcomes-learning-objectives}{%
\subsection{Outcomes / Learning
Objectives}\label{outcomes-learning-objectives}}

\begin{itemize}
\tightlist
\item
  Students will learn about classification
\item
  How classifications is used in Machine learning (ML)
\item
  How to create their own ml algorithm
\item
  Create their own application
\item
  Be introduced to computer science.
\end{itemize}

\hypertarget{students-will}{%
\subsubsection{Students will:}\label{students-will}}

\begin{itemize}
\tightlist
\item
  Explain that machine learning is when computers detect patterns
\item
  Make their own rules (a model) for describing those patterns
\item
  Train a machine learning model using Teachable Machine
\item
  Use conditional statements
\end{itemize}

\hypertarget{prep}{%
\subsection{Prep}\label{prep}}

\begin{longtable}[]{@{}
  >{\raggedright\arraybackslash}p{(\columnwidth - 2\tabcolsep) * \real{0.5500}}
  >{\raggedright\arraybackslash}p{(\columnwidth - 2\tabcolsep) * \real{0.4500}}@{}}
\toprule\noalign{}
\begin{minipage}[b]{\linewidth}\raggedright
Item
\end{minipage} & \begin{minipage}[b]{\linewidth}\raggedright
Qty
\end{minipage} \\
\midrule\noalign{}
\endhead
\bottomrule\noalign{}
\endlastfoot
\href{https://ejboettcher.github.io/gemcityML-No-CodeAI/data/monkey_class.pdf}{Monkey
Carts Printed} & 1 set per group \\
laptop with web camera & 1 per group \\
Internet & \\
Pen and Paper & 1 per student \\
\end{longtable}

\hypertarget{lesson}{%
\section{Lesson}\label{lesson}}

Outline:

\begin{itemize}
\tightlist
\item
  Classes and Models (No computers, Need monkey cards)
\item
  \href{https://ejboettcher.github.io/gemcityML-No-CodeAI/application_demo/index.html}{Finished
  Application Demo}
\item
  Walk through \href{https://teachablemachine.withgoogle.com/}{Teachable
  Machines}
\item
  Student build their own application (two class AI model)
\item
  (stretch) Students build three class application
\item
  (stretch) Students build a nicer application
\end{itemize}

\hypertarget{opening-15-min}{%
\subsection{Opening (15 min)}\label{opening-15-min}}

\textbf{HOOK}\\
Show finished
\href{https://ejboettcher.github.io/gemcityML-No-CodeAI/application_demo/index.html}{Application
Demo}

\textbf{Ask:} How does that work?

Walk through what a class is.

Give students a set of the \textbf{green} monkey cards (from AI
Unplugged). Have teams divide their chart paper into 2 classes: Biting
and Non Biting.

\emph{Training data} (blue paper):

\begin{itemize}
\tightlist
\item
  biting: 1, 2, 3, 4
\item
  non- biting: 5-20 Have them decide which characteristics are for
  biting monkeys. This is done as a group.
\end{itemize}

Then show them the test data (green paper) and see how well their model
did.

\emph{Test data} (green paper)

\begin{itemize}
\tightlist
\item
  Biting: 22, 23, 24
\item
  Non-biting: 21, 25 - 40
\end{itemize}

\href{https://www.aiunplugged.org/monkey_game.pdf}{AI Unplugged} has
more example in this \href{./data/monkey_paper.pdf}{paper}

\hypertarget{ml-explained-2.5-min}{%
\subsection{ML Explained (2.5 min)}\label{ml-explained-2.5-min}}

Overview Video on Machine Learning (\textasciitilde{} 2 minutes)

\href{https://www.youtube.com/watch?v=f_uwKZIAeM0}{YouTube} (very simple
explanation)

\hypertarget{train-model-10-min}{%
\subsection{Train Model (10 min)}\label{train-model-10-min}}

Train Model with Teachable Machines.

\begin{itemize}
\tightlist
\item
  Demo how to train a model on Teachable Machine
\item
  Give students 6-7 minutes to train their own.

  \begin{itemize}
  \tightlist
  \item
    Have students go to
    \href{https://teachablemachine.withgoogle.com/}{Teachable Machine}
  \item
    Click \emph{Get Started} and start an image classification
  \item
    Let students create two class classification for any school
    acceptable hand jester.

    \begin{itemize}
    \tightlist
    \item
      Keep the images simple
    \item
      name your classes something descriptive: Cat / Dog
    \item
      Ask how you could account for differences: skin color, jewelry,
      nail color.
    \end{itemize}
  \end{itemize}
\end{itemize}

\hypertarget{run-models-10min}{%
\subsection{Run Models (10min)}\label{run-models-10min}}

\begin{itemize}
\tightlist
\item
  Download model.

  \begin{itemize}
  \tightlist
  \item
    Show students how to copy their model to a folder (static) in
    student\_application\_start,
  \item
    Update the URL in the \texttt{my\_model.js} (line 5)
  \end{itemize}
\item
  (stretch) Show students how to add an image to the first ``if''
  condition. (on line 64) (hint: look at the \texttt{application\_demo}
  folder)

  \begin{itemize}
  \tightlist
  \item
    Use Wikipedia images search for emojis
  \end{itemize}
\item
  (stretch) Ask how the Javascript syntax is different than the Python
  Syntax
\end{itemize}

\hypertarget{closing}{%
\subsection{Closing}\label{closing}}

\begin{itemize}
\tightlist
\item
  Have each group Demo their application
\item
  Student Reflection:

  \begin{itemize}
  \tightlist
  \item
    How could you use ML application in your school, home, car?
  \item
    What would you have to consider when training a model?
  \end{itemize}
\item
  Celebrate: You created a working ML models!\\
\item
  Follow-up Resources:

  \begin{itemize}
  \tightlist
  \item
    \href{https://www.aiunplugged.org/monkey_game.pdf}{AI Unplugged}
  \item
    AI for ALl summer programs
  \item
    \href{https://thecodingtrain.com/}{The Code Train}
  \item
    \href{https://codelabs.developers.google.com/tensorflowjs-transfer-learning-teachable-machine\#6}{Google
    Tutorial}
  \end{itemize}
\end{itemize}



\end{document}
